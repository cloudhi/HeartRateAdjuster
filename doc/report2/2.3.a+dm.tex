    Our system utilizes a three-tier architecture system and consists of 3 layers. These include a presentation tier, an application tier, and a data tier. Our presentation layer is primarily represented by our mobile interface which is used to display our application’s relevant information. It also allows the user to interact with our system by inputting commands and accepting outputs. Meanwhile, our application layer consists of logical operations and data access. For example, our song-selection algorithm would be included in this layer. This application layer uses logical operations to convert raw user data into readable results. Finally, our data tier consists of our database where our information is stored and retrieved. \\

    These three tiers are separated from each other to allow for encapsulation and data abstraction. We want each tier to hide its usage from implementation and to preserve the integrity of our data. We also want to reduce the overall complexity of our system. However, each tier must maintain a sufficient level of communication and be able to retrieve needed data from each other. In a common scenario for our system, our application layer may request information from the data tier. It then processes this information and returns it to the presentation tier in response to the user request. A visual diagram was provided in our earlier stage of planning in the section titled System Architecture Diagram.
