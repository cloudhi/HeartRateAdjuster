\documentclass[letterpaper,english, 12pt]{scrreprt}
\usepackage[T1]{fontenc}
\usepackage{array}
\usepackage{graphicx}
\usepackage{float}
\usepackage{titlepic}
\usepackage{multirow}
\usepackage[bookmarks=true]{hyperref}
\newcolumntype{C}[1]{>{\centering\let\newline\\\arraybackslash\hspace{0pt}}m{#1}}

\begin{document}

\begin{center}
        \begin{tabular}{|C{15cm}|}
                \hline
                        \textbf{Login/Log out}\\
                \hline
                        \begin{flushleft}
                                \textbf{Tests: } Data Manager, Graphical User Interface
                        \end{flushleft}
                        \begin{flushleft}
                                \textbf{Assumption: } The program has displayed the login page and is waiting for the user's input
                        \end{flushleft}
                        \begin{flushleft}
                                \textbf{Steps:}
                        \end{flushleft}
				\begin{itemize}
					\item Input a valid user account and corresponding password.
					\item $\rightarrow$ Click "login button"
					\item $\rightarrow$ if Step 2 succeeds, click "log out" button
				\end{itemize}
			\begin{flushleft}
				\textbf{Expected: } User successfully logs in and logs out
			\end{flushleft}
                        \begin{flushleft}
                                \textbf{Fails if: }
                        \end{flushleft}
                                \begin{itemize}
                                        \item The user fails to login
					\item The user fails to log out
                                \end{itemize}
				\\
		\hline
        \end{tabular}
\end{center}

\begin{center}
        \begin{tabular}{|C{15cm}|}
                \hline
                        \textbf{Increase/Decrease Target Heart Rate}\\
                \hline
                        \begin{flushleft}
                                \textbf{Tests: } Graphical User Interface
                        \end{flushleft}
                        \begin{flushleft}
                                \textbf{Assumption: } The user has successfully logged in, and is on the correct screen
                        \end{flushleft}
                        \begin{flushleft}
                                \textbf{Steps:}
                        \end{flushleft}
				\begin{itemize}
					\item Press the button to increase target heart rate
					\item $\rightarrow$ If the target heart rate has displayed an incremental change, press the button to decrease target heart rate
				\end{itemize}
			\begin{flushleft}
				\textbf{Expected: } Target heart rate is successfully incremented/decremented when the correct buttons are pressed
			\end{flushleft}
                        \begin{flushleft}
                                \textbf{Fails if: }
                        \end{flushleft}
                                \begin{itemize}
                                        \item Target heart rate does not change
					\item Target heart rate changes in incorrect direction
                                \end{itemize}
				\\
		\hline
        \end{tabular}
\end{center}

\begin{center}
        \begin{tabular}{|C{15cm}|}
                \hline
                        \textbf{Start/Pause Workout}\\
                \hline
                        \begin{flushleft}
                                \textbf{Tests: } Graphical User Interface
                        \end{flushleft}
                        \begin{flushleft}
                                \textbf{Assumption: } User is wearing the heart rate monitor and has selected a target heart rate
                        \end{flushleft}
                        \begin{flushleft}
                                \textbf{Steps:}
                        \end{flushleft}
				\begin{itemize}
					\item User presses button to begin workout
					\item $\rightarrow$ If the workout successfully begins, press button to pause workout
				\end{itemize}
			\begin{flushleft}
				\textbf{Expected: } When the user presses the button to begin the workout, the workout will begin. When the user presses the button pause the workout, the workout will pause.
			\end{flushleft}
                        \begin{flushleft}
                                \textbf{Fails if: }
                        \end{flushleft}
                                \begin{itemize}
                                        \item The workout does not begin when the button is pressed
					\item The workout does not pause when the button is pressed
                                \end{itemize}
				\\
		\hline
        \end{tabular}
\end{center}

\begin{center}
        \begin{tabular}{|C{15cm}|}
                \hline
                        \textbf{Skip Track}\\
                \hline
                        \begin{flushleft}
                                \textbf{Tests: } Graphical User Interface
                        \end{flushleft}
                        \begin{flushleft}
                                \textbf{Assumption: } User has already begun workout and a song is currently playing
                        \end{flushleft}
                        \begin{flushleft}
                                \textbf{Steps:}
                        \end{flushleft}
				\begin{itemize}
					\item User presses the button to skip the current track
				\end{itemize}
			\begin{flushleft}
				\textbf{Expected: } The application will play a new song
			\end{flushleft}
                        \begin{flushleft}
                                \textbf{Fails if: }
                        \end{flushleft}
                                \begin{itemize}
					\item Pressinig the button does not play the next  song
                                \end{itemize}
				\\
		\hline
        \end{tabular}
\end{center}

\begin{center}
        \begin{tabular}{|C{15cm}|}
                \hline
                        \textbf{Display Graphs}\\
                \hline
                        \begin{flushleft}
                                \textbf{Tests: } Graphical User Interface
                        \end{flushleft}
                        \begin{flushleft}
                                \textbf{Assumption: } User has logged data into the application and is on the correct screen
                        \end{flushleft}
                        \begin{flushleft}
                                \textbf{Steps:}
                        \end{flushleft}
				\begin{itemize}
					\item User presses the button to display statistics
				\end{itemize}
			\begin{flushleft}
				\textbf{Expected: } The application will display graphs for the
			\end{flushleft} user
                        \begin{flushleft}
                                \textbf{Fails if: }
                        \end{flushleft}
                                \begin{itemize}
                                        \item The user presses the button and graphs do not display
                                \end{itemize}
				\\
		\hline
        \end{tabular}
\end{center}

\begin{center}
        \begin{tabular}{|C{15cm}|}
                \hline
                        \textbf{Set Media Library Location}\\
                \hline
                        \begin{flushleft}
                                \textbf{Tests: } Graphical User Interface, Data Manager
                        \end{flushleft}
                        \begin{flushleft}
                                \textbf{Assumption: } The user has logged in successfully and is on the correct screen
                        \end{flushleft}
                        \begin{flushleft}
                                \textbf{Steps:}
                        \end{flushleft}
				\begin{itemize}
					\item User presses button to change media library location
					\item User locates the directory where media is stored
					\item User chooses the correct directory which contains media
					\item User begins workout
				\end{itemize}
			\begin{flushleft}
				\textbf{Expected: } The application will play media from the correct chosen directory
			\end{flushleft}
                        \begin{flushleft}
                                \textbf{Fails if: }
                        \end{flushleft}
                                \begin{itemize}
                                        \item Media is played from an incorrect directory
                                \end{itemize}
				\\
		\hline
        \end{tabular}
\end{center}
\end{document}
